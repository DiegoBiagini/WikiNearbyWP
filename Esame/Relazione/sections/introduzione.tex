\section*{Introduzione}
Il progetto che è stato realizzato è un plugin per il CMS Wordpress, che abbiamo scelto di chiamare WikiNearby.\\
Lo scopo di questo plugin è quello di offrire agli sviluppatori di un sito web incentrato sul turismo la possibilità di inserire un widget dentro a post o pagine Wordpress per visualizzare i luoghi di interesse culturale vicini ad un luogo geografico inserito a priori.\\\\
Per poter recuperare le informazioni circa i luoghi di interesse sarà consultato il database di articoli Wikipedia, questo è fatto attraverso una particolare API di Wikipedia, GeoData, la quale si occupa principalmente di organizzare certe pagine(per esempio di monumenti,musei, etc.) assegnandogli delle coordinate.\\
Oltre a questo permette di consultare informazioni sui luoghi vicini a una coordinata geografica inviata dall'utente.\\\\
Le tecnologie usate per la realizzazione di questo applicativo sono tutte quelle trattate a lezione, ovvero PHP e Wordpress API per realizzare l'interfacciamento tra utente Wordpress e plugin, nonchè HTML5 e CSS3 per la visualizzazione dell'applicativo e infine Javascript, JQuery e AJAX per rendere interattivo lo strumento e per recuperare le informazioni necessarie da Wikipedia.\\\\
Il lavoro è stato diviso principalmente in due parti: 
\begin{itemize}
    \item Realizzazione di una pagina "statica" che adempie al suo obiettivo a fronte di un luogo definito, cioè visualizzare i luoghi di interesse vicino ad esso
    \item Realizzazione dell'infrastruttura di plugin Wordpress che consente al proprietario di un sito di aggiungere molteplici luoghi, per poter poi visualizzare lo widget applicato su quel particolare luogo
\end{itemize}
Infine le due parti sono state fuse insieme per rendere il plugin funzionante.